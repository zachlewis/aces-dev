% This file contains the content for a main section
\regularsectionformat	% Change formatting to that of "Introduction" section
%% Modify below this line %%
\chapter{Specification}

\section{Naming conventions}
The quasi-logarithmic encoding of ACES specified in \autoref{sec:ACEScct} shall be known as ACEScct.

\section{Color component value encoding}
ACEScct values are encoded as 32-bit floating-point numbers. This floating-point encoding uses 32 bits per component as described in IEEE 754.

\section{Color space chromaticities}
\label{sec:colorspace}
ACEScct uses a different set of primaries than ACES RGB primaries defined in SMPTE ST 2065-1. The CIE 1931 colorimetry of the ACEScct RGB primaries and white are specified below.

\subsection{Color primaries}
The RGB primaries chromaticity values, known as AP1, shall be those found in \autoref{table:AP1rgb}.

\begin{center}
\begin{tabularx}{4.5in}{XlllXll}
        & R       & G       & B       & & CIE x & CIE y \\ \hline
Red     & 1.00000 & 0.00000 & 0.00000 & & 0.713 & 0.293 \\
Green   & 0.00000 & 1.00000 & 0.00000 & & 0.165 & 0.830 \\
Blue    & 0.00000 & 0.00000 & 1.00000 & & 0.128 & 0.044 \\
\end{tabularx}
\captionof{table}{ACEScct RGB primaries chromaticity values}    
\label{table:AP1rgb}
\end{center}

\subsection{White Point}
The white point shall be that found in \autoref{table:AP1w}.

\begin{center}
\begin{tabularx}{4.5in}{XlllXll}
        & R       & G       & B       & & CIE x & CIE y \\ \hline
White   & 1.00000 & 1.00000 & 1.00000 & & 0.32168 & 0.33767 \\
\end{tabularx}
\captionof{table}{ACEScct RGB white point chromaticity values}    
\label{table:AP1w}
\end{center}

\newpage
\section{ACEScct}
\label{sec:ACEScct}
The following functions shall be used to convert between ACES values, encoded according to SMPTE ST 2065-1, and ACEScct.

\subsection{Encoding Function}
ACES $R$, $G$, and $B$ values shall be converted to $lin_{AP1}$ $R$, $G$, and $B$ values using the transformation matrix ($TRA_{1}$) calculated and applied using the methods provided in Section 4 of SMPTE RP 177:1993.

$lin_{AP1}$ $R$, $G$, and $B$ values shall be converted to ACEScct values using Equation \ref{eq:linAP12ACEScct}.

\begin{floatequ} 
\begin{equation} 
    ACEScct = \left\{ 
    \begin{array}{l l }
        10.5402377416545 \times lin_{AP1} + 0.0729055341958355;    & \quad lin_{AP1} \leq 0.0078125 \\[10pt]
        \dfrac{\log_{2}(lin_{AP1}) + 9.72}{17.52}; & \quad lin_{AP1} > 0.0078125 \\    
    \end{array} \right.
\end{equation}
\caption{lin\textsubscript{AP1} to ACEScct}
\label{eq:linAP12ACEScct}
\end{floatequ}

\note{Equation \ref{eq:ACES2linAP1} shows the relationship between ACES $R$, $G$, and $B$ values and $lin_{AP1}$ $R$, $G$, and $B$ values. $TRA_{1}$, rounded to 10 significant digits, is derived from the product of $NPM_{AP1}$ inverse and $NPM_{AP0}$ calculated using methods provided in Section 3.3 of SMPTE RP 177:1993. AP0 are the primaries of ACES specified in SMPTE ST 2065-1. AP1 are the primaries of ACEScct specified in \autoref{sec:colorspace}.}

\begin{floatequ} 
\begin{gather}
    \begin{bmatrix}
        R_{lin_{AP1}}\\
        G_{lin_{AP1}}\\
        B_{lin_{AP1}}
    \end{bmatrix}
    =
    TRA_{1}
    \cdot
    \begin{bmatrix}
        R_{ACES}\\
        G_{ACES}\\
        B_{ACES}
    \end{bmatrix} \\
    \\
    TRA_{1} =
    \begin{bmatrix*}[r]
        1.4514393161 & -0.2365107469 & -0.2149285693 \\
       -0.0765537734 &  1.1762296998 & -0.0996759264 \\
        0.0083161484 & -0.0060324498 &  0.9977163014 \\
    \end{bmatrix*} \\
    \\
    TRA_{1} = NPM^{-1}_{AP1} \cdot NPM_{AP0}
\end{gather}
\caption{ACES to lin\textsubscript{AP1}}
\label{eq:ACES2linAP1}
\end{floatequ}

\subsection{Decoding Function}
ACEScct $R$, $G$, and $B$ values shall be converted to $lin_{AP1}$ values using Equation \ref{eq:ACEScct2linAP1}.

\begin{floatequ} 
\begin{equation}
\resizebox{\textwidth}{!}{$
    lin_{AP1} = \left\{ 
    \begin{aligned}
        &\dfrac{\left(ACEScct-0.0729055341958355\right)}{10.5402377416545};& ACEScct& \leq 0.155251141552511 \\[10pt]
        &2^{(ACEScct \times 17.52-9.72)}; &0.155251141552511 \leq ACEScct& < \dfrac{\log_{2}(65504)+9.72}{17.52} \\[10pt]
        &65504; & ACEScct& \geq \dfrac{\log_{2}(65504)+9.72}{17.52} \\    
    \end{aligned} \right.
$}
\end{equation}
\caption{ACEScct to lin\textsubscript{AP1}}
\label{eq:ACEScct2linAP1}
\end{floatequ}

$lin_{AP1}$ $R$, $G$, and $B$ values shall be converted to ACES $R$, $G$, and $B$ values using the transformation matrix ($TRA_{2}$) calculated and applied using the methods provided in Section 4 of SMPTE RP 177:1993.

\note{Equation \ref{eq:linAP12ACES} shows the relationship between ACES $R$, $G$, and $B$ values and ACEScct $R$, $G$, and $B$ values. $TRA_{2}$, rounded to 10 significant digits, is derived from the product of $NPM_{AP0}$ inverse and $NPM_{AP1}$ calculated using methods provided in Section 3.3 of SMPTE RP 177:1993. AP0 are the primaries of ACES specified in SMPTE ST 2065-1. AP1 are the primaries of ACEScct specified in \autoref{sec:colorspace}.}

\begin{floatequ} 
\begin{gather}
    \begin{bmatrix}
        R_{ACES}\\
        G_{ACES}\\
        B_{ACES}
    \end{bmatrix}
    =
    TRA_{2}
    \cdot
    \begin{bmatrix}
        R_{lin_{AP1}}\\
        G_{lin_{AP1}}\\
        B_{lin_{AP1}}
    \end{bmatrix} \\
    \\
    TRA_{2} =
    \begin{bmatrix*}[r]
        0.6954522414 & 0.1406786965 & 0.1638690622 \\
        0.0447945634 & 0.8596711185 & 0.0955343182 \\
        -0.0055258826 & 0.0040252103 & 1.0015006723 \\
    \end{bmatrix*} \\
    \\
    TRA_{2} = NPM^{-1}_{AP1} \cdot NPM_{AP0}
\end{gather}
\caption{lin\textsubscript{AP1} to ACES}
\label{eq:linAP12ACES}
\end{floatequ}