% This section contains the content for the Terms and Definitions
\numberedformat
\chapter{Terms and Definitions}
The following terms and definitions are used in this document.
%% Modify below this line %%

\term{Pre-grade}
Preliminary color adjustment (``grade'') applied after image creation; typically used for balancing exposure and color for later use in production.

\term{ClipName}
Name attached to an image sequence or segment of an image sequence; often used in EDLs to describe the version of images used in the edited sequence.  

\term{Source\_Media\_ID}
General term for identifying the original source name of images or image sequences when the images/sequences were created; often referenced in EDLs. Other terms used for this purpose include Tape Name (CMX EDL formats) and ReelName. An example Source\_Media\_ID name is \textit{A004B003}, which describes \textit{Camera `A' magazine 004  clip B  003}. Noted: for international usage, the Source\_Media\_ID must be a Unicode string.

\term{DateTime}
(reference: ISO8601:2004) timestamp format  

The DateTime is specified in the following form "YYYY-MM-DDThh:mm:ss\{offset\}" where:
\begin{itemize}
    \item YYYY indicates the year
    \item MM indicates the month
    \item DD indicates the day
    \item T indicates the start of the required time section
    \item hh indicates the hour
    \item mm indicates the minute
    \item ss indicates the second
    \item \{offset\} time zone offset from UTC
\end{itemize}
\note{All components are required.}

Example:  2014-11-20T12:24:13-8:00

\term{TransformID}
String identifying the ACES transform, generally a Color Transformation Language (CTL) file or LUT file. Please see the ACES System Versioning Specification for more information on the format to use for TransformIDs.