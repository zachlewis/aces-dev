\documentclass[10pt]{academydoc}
\pagestyle{plain}

% Set Document Details
\doctype{proc} % spec, proc, tb (Specification, Procedure, Technical Bulletin)
\docname{Recommended Procedures for the Creation and Use of Digital Camera System Input Device Transforms (IDTs)}
\altdocname{Creation and Use of Digital Camera System Input Device Transforms (IDTs)}
% Sets the document name used in header - usually an abbreviated document title
\docnumber{P-2013-001}
\committeename{Academy Color Encoding System (ACES) Project Committee}
\docdate{March 29, 2016}
\summary{
In the Academy Color Encoding System, an Input Device Transform (IDT) processes non-color-rendered RGB image values from a digital camera system's capture of a scene lit by an assumed illumination source (the scene adopted white). The results of this process are white-balanced ACES RGB relative exposure values.

Camera system vendors are recommended to provide two IDTs for each product, one optimized for CIE Illuminant D55 (daylight) and a second optimized for the ISO 7589 Studio Tungsten illuminant. Camera system vendors may optionally provide additional IDTs for common illumination sources such as Hydrargyrum Medium-arc Iodide (HMI) and KinoFlo\textsuperscript{\textregistered{}} lamps.

The main body of this document provides a procedure for the creation and use of an IDT from the measured spectral responsivities of a digital camera system. Appendices provide examples of the use of the procedure, each illustrating a different engineering tradeoff. An additional appendix provides an alternative procedure for IDT creation if the spectral data required for the recommended procedure are unobtainable.
}

% Document Starts Here
\begin{document}

\maketitle

% This file contains the content for the Notices
\prelimsectionformat	% Change formatting to that of "Notices" section
\chapter[Notices]{\uppercase{Notices}}
%% Modify below this line %%

\copyright\the\year{} Academy of Motion Picture Arts and Sciences (A.M.P.A.S.). All rights reserved. This document is provided to individuals and organizations for their own internal use, and may be copied or reproduced in its entirety for such use. This document may not be published, distributed, publicly displayed, or transmitted, in whole or in part, without the express written permission of the Academy.

The accuracy, completeness, adequacy, availability or currency of this document is not warranted or guaranteed. Use of information in this document is at your own risk. The Academy expressly disclaims all warranties, including the warranties of merchantability, fitness for a particular purpose and non-infringement.

Copies of this document may be obtained by contacting the Academy at councilinfo@oscars.org.

``Oscars,'' ``Academy Awards,'' and the Oscar statuette are registered trademarks, and the Oscar statuette a copyrighted property, of the Academy of Motion Picture Arts and Sciences.

% This paragraph is optional.  Comment out if you wish to remove it.
This document is distributed to interested parties for review and comment. A.M.P.A.S. reserves the right to change this document without notice, and readers are advised to check with the Council for the latest version of this document.

% This paragraph is optional.  Comment out if you wish to remove it.
The technology described in this document may be the subject of intellectual property rights (including patent, copyright, trademark or similar such rights) of A.M.P.A.S. or others. A.M.P.A.S. declares that it will not enforce any applicable intellectual property rights owned or controlled by it (other than A.M.P.A.S. trademarks) against any person or entity using the intellectual property to comply with this document.

% This paragraph is optional.  Comment out if you wish to remove it.
Attention is drawn to the possibility that some elements of the technology described in this document, or certain applications of the technology may be the subject of intellectual property rights other than those identified above. A.M.P.A.S. shall not be held responsible for identifying any or all such rights. Recipients of this document are invited to submit notification to A.M.P.A.S. of any such intellectual property of which they are aware.

\vspace{10pt}
These notices must be retained in any copies of any part of this document. \newpage
% This file contains the content for the Revision History and 
\prelimsectionformat	% Change formatting to that of "Notices" section
\chapter{Revision History}
%% Modify below this line %%

\begin{tabularx}{\linewidth}{|l|l|X|}
    \hline
    Version & Date       & Description \\ \hline
    1.0     & 12/19/2014 & Initial Version
    \\ \hline
    1.0.1   & 04/24/2015 & Formatting and typo fixes \\ \hline
            & 03/29/2016 & Remove version number - to use modification date as UID \\ \hline
    &   &   \\ \hline
    &   &   \\ \hline
    &   &   \\ \hline
\end{tabularx}

\vspace{0.25in} % <-- DO NOT REMOVE
\chapter{Related Academy Documents} % <-- DO NOT REMOVE
\begin{tabularx}{\linewidth}{|l|X|}
    \hline
    Document Name & Description \\ \hline
    S-2008-001 & Academy Color Encoding Specification (ACES) \\ \hline
    & \\ \hline
    & \\ \hline
    & \\ \hline
    & \\ \hline
\end{tabularx} \newpage

\tableofcontents \newpage

% This file contains the content for the Acknowledgements
\cleardoublepage
\unnumberedformat	% Change formatting to that of "Acknowledgements" section
\chapter{Acknowledgements} 	% Do not modify section title
%% Modify below this line %%

The Science and Technology Council wishes to acknowledge the following key contributors to the drafting of this document.

\begin{center}
    \begin{tabular}{llll}
        Joseph Goldstone & Jack Holm & Edward Giorgianni & Alex Forsythe \\
        Lars Borg & Harald Brendel & Jim Houston
    \end{tabular}
\end{center}
    
%The Council also wishes to acknowledge the contributions of the members of the \Committeename{} for the development of the concepts and technologies that led to the publication of this document.
%
%\begin{center}
%    \Committeechair{}, Chair, \Committeename{}
%\end{center} \newpage
% This file contains the content for the Introduction
\unnumberedformat	    % Change formatting to that of "Introduction" section
\chapter{Introduction} 	% Do not modify section title
%% Modify below this line %%

The Academy Color Encoding System is a free, open, device-independent color management and image interchange system that can be applied to almost any current or future workflow. It was developed by hundreds of the industry's top scientists, engineers, and end users, working together under the auspices of the Academy of Motion Picture Arts and Sciences.

The primary color encoding in the Academy Color Encoding System (ACES) is the Academy Color Encoding Specification (ACES2065-1).  Academy Color Encoding Specification is standardized in SMPTE ST 2065-1:2012 \cite{SMPTE20651}.  As part of the specification, the encoding primaries and white point were specified as CIE xy chromaticity coordinates to allow for the transformation of ACES2065-1 RGB values to and from other color spaces including CIE XYZ.  Though the CIE xy chromaticity coordinates of encoding red, green, blue and white primaries are only one factor important to unambiguous color interchange\cite{giorgianni}, their specification is required for the calculation of a normalized primary matrix used in color space transformations \cite{smpteRP1997}. The white point used in ACES2065-1 was later adopted for use in other ACES encodings such as ACEScg, ACEScc, ACEScct, etc \cite{ACEScg,ACEScc,ACEScct}. For brevity and inclusiveness, the white point used in the various encodings will be referred to as "the ACES white point" throughout the remainder of this document unless more specificity is required.

The derivation of the ACES white point chromaticity coordinates outlined in this document is intended to help technical users of the ACES system calculate transformations to and from the various ACES encodings in as accurate a manner as possible.  The white point of the ACES encodings does not limit the choice of sources that may be used to photograph or generate source images, nor does it dictate the white point of the reproduction. Using various techniques beyond the scope of this document, the chromaticity of the reproduction of equal ACES2065-1 red, green and blue values (ACES2065-1 \rgbequal) may match the chromaticity of the ACES white point, the display calibration white point, or any other white point preferred for technical or aesthetic reasons

ACES technical documentation is available via ACEScentral.com and oscars.org/aces for product developers wishing to implement ACES concepts and specifications into their products and for workflow/pipeline designers to use ACES concepts and ACES-enabled products for their productions.

% This file contains the content for the Scope
\cleardoublepage
\numberedformat	
\chapter{Scope} 	% Do not modify section title
%% Modify below this line %%

This document describes the derivation of the ACES white point CIE chromaticity coordinates and details of why the chromaticity coordinates were chosen.  This document includes links to an example Python implementation of the derivation and an iPython notebook intended to help readers reproduce the referenced values.

This document is primarily intended for those interested in understanding the details of the technical specification of ACES and the history of its development. The definition of a color space encoding's white point chromaticity coordinates is one important factor in the definition of a color managed system. The white point used in various ACES encodings does not dictate the creative white point of images created or mastered using the ACES system. It exists to enable accurate conversion to and from the other color encodings such as CIE XYZ.  The proper usage of the ACES white point in conversion, mastering, or reproduction are beyond the scope of this document.  For example, the proper usage of the ACES white point in encoding scene colorimetry in ACES2065-1 is detailed in P-2013-001 \cite{idt}.  

% This section contains the content for the References
\numberedformat
\chapter{References}
The following standards, specifications, articles, presentations, and texts are referenced in this text:
%% Modify below this line %%

SMPTE ST 2065-1:2012, Academy Color Encoding Specification (ACES)

SMPTE RP 177:1993, Derivation of Basic Television Color Equations
% This section contains the content for the Terms and Definitions
\numberedformat
\chapter{Terms and Definitions}
The following terms and definitions are used in this document.
%% Modify below this line %%

\term{Academy Color Encoding Specification (ACES)}
RGB color encoding for exchange of image data that have not been color rendered, between and throughout production and postproduction, within the Academy Color Encoding System. ACES is specified in SMPTE ST 2065-1.

\term{ACES RGB relative exposure values}
Relative responses to light of the ACES Reference Image Capture Device, determined by the integrated spectral responsivities of its color channels and the spectral radiances of scene stimuli.

\term{ACES unity neutral}
A triplet of ACES RGB relative exposure values all of which have unity magnitude.

\term{chromatic adaptation}
A process by which the visual mechanism adjusts in response to the radiant energy to which the eyes are exposed.

\term{chromaticity}
A property of a color stimulus defined by the ratios of each tristimulus value of the color stimulus to their sum.

\term{color rendering}
The mapping of image data representing the color-space coordinates of the elements of a scene to output-referred image data representing the color-space coordinates of the elements of a reproduction. Color rendering generally consists of one or more of the following: compensating for differences in the input and output viewing conditions, tone scale and gamut mapping to map the scene colors onto the dynamic range and color gamut of the reproduction, and applying preference adjustments.

\term{color stimulus}
Radiant energy such as that produced by an illumination source, by the reflection of light from a reflective object, or by the transmission of light through a transmissive object, or a combination of these.

\term{focal-plane-referred}
A representation of a captured scene that includes any flare light introduced by the camera’s optical system.

\term{Input Device Transform (IDT)}
A signal-processing transform that maps an image capture system’s representation of an image to ACES RGB relative exposure values.

\term{memory color}
A color sensation derived from memory rather than the immediate perception of a color stimulus.

\term{radiometric linearity}
An attribute of a representation of measured energy in which a change in the amount of measured energy is accompanied by an equal change in the representation of that energy, e.g. a doubling of measured energy is matched by a doubling of the quantity representing that energy.

\term{Reference Input Capture Device (RICD)}
A hypothetical camera, which records an image of a scene directly as ACES RGB relative exposure values.

\term{re-illumination}
An alteration of colors of a captured scene simulating the reflectance of objects in a scene illuminated by an illumination source other than the one under which the scene was captured.

\term{scene adopted white}
A spectral radiance distribution as seen by an image capture or measurement device that is converted to color signals that are considered to be perfectly achromatic and to have an observer adaptive luminance factor of unity; i.e. color signals that are considered to correspond to a perfect white diffuser.

\term{spectral responsivity}
The response of a detection system as a function of wavelength.

\term{spectral sensitivity}
The response of a detector to monochromatic stimuli of equal radiant power.

\term{white balance}
The process of adjusting the RGB signals of an electronic camera system such that equal signals are produced for an object in the scene that is desired to be achromatic \newpage
% This file contains the content for the Procedure
\regularsectionformat
\chapter{Recommended procedure for IDT creation}
%% Modify below this line %%

\label{chap:procedure}

The core of a digital camera system IDT is the 3x3 matrix by which radiometrically linear camera RGB code values are transformed to produce ACES RGB relative exposure values. This procedure describes the derivation of such a matrix.

To have fully followed the recommendations given in this document, IDT authors should provide an IDT for daylight and an IDT for tungsten illumination sources. The daylight illumination source should have the spectral power distribution of CIE Standard Illuminant D55; the tungsten illumination source, ISO 7589 Studio Tungsten.

\note{All references to CIE XYZ tristimulus values in this recommendation refer to colorimetric coordinates defined with respect to the CIE 1931 XYZ color space and the CIE 1931 2$^\circ$ Standard Colorimetric Observer.}

\section{Selection and Weighting of Training Data}
The quality of an IDT constructed as per this recommendation will depend heavily on the diversity and distribution of the set of spectral radiances used in IDT matrix optimization. Training spectra selection should attempt to sample the entire space of real-world scene spectra. Particularly important scene spectra, such as memory color spectra, may be weighted more heavily than colors rarely found in everyday life.

Training spectra should represent in situ stimuli if possible: the training spectrum representing a blade of grass illuminated by sunlight should be representative of that blade of grass in the context of surrounding leaves of grass, rather than in isolation on a laboratory slide, since very few motion pictures feature grass examined under a microscope. Only a small part of the reflected light from a single blade of grass is a first-bounce reflection of sunlight; the larger part of the illumination of a blade of grass in production photography will bounce off of or pass through neighboring blades of grass.

\section{Assumptions and prerequisites}
This procedure is appropriate when the IDT manufacturer desires to implement some form of neutral chromaticity difference compensation. It presumes the following information is available to the IDT manufacturer before applying the procedure:

\begin{itemize}
	\item	A set of training spectra against which the IDT matrix will be optimized, expressed either as spectral radiances, or as spectral reflectances which can be multiplied with the spectral power distribution of the scene adopted white to form spectral radiances.
	\item	The spectral power distribution of the scene adopted white.
	\item	A neutral chromaticity difference compensation strategy when the scene adopted white chromaticity differs from that of ACES neutrals.
	\item	A transform from camera system image data code values to three-channel radiometrically linear image data.
	\item	The spectral responsivities of the digital camera system for which the IDT is being created.
	\item	A transform from CIE XYZ colorimetry into an error minimization color space.
	\item	The error metric (cost function) that will be minimized, and any weights that will be applied differentially to the training spectra.
\end{itemize}

The sampling bounds and sampling interval of the spectra for the scene adopted white, for the RICD spectral responsivities, for the camera system spectral responsivities and for the training spectra are assumed to be the same so that mathematical operations can be performed without ambiguity. In addition, the procedure presumes the IDT author has access to software capable of mathematical operations including vector and matrix algebra, linear regression and minimization with user-defined cost functions.

\section{Symbols}
\label{sec:symbols}
In the table and equations below, lower case italic letters ($m$) represent scalars, lower case bold face letters ($\mathbf{h}$) represent vectors, and upper case bold face letters ($\mathbf{M}$) represent matrices.

{\renewcommand{\arraystretch}{1.25}
\begin{tabularx}{\textwidth}{|l|l|X|}
	\hline
	\bfseries{Symbol} & \bfseries{Domain} & \bfseries{Description} \\ \hline
	$m$ & $\Re$ & Number of samples in spectral data. \\ \hline
	$n$ & $\Re$ & Number of training spectra. \\ \hline
	$\mathbf{C}$ & $\Re^{m\times3}$ & Spectral responsivities of camera system (with channel gain differences maintained). \\ \hline
	$\big(\mathbf{r}_{i}\big)^{n}_{i=1}$ & $\Re^m$ & Spectral reflectances of training spectra. \\ \hline
	$\big(\mathbf{t}_{i}\big)^{n}_{i=1}$ & $\Re^m$ & Spectral radiances of training spectra. \\ \hline
	$\big(\mathbf{x}_{i}\big)^{n}_{i=1}$ & $\Re^3$ & CIE XYZ tristimulus values of training spectra. \\ \hline
	$\big(\mathbf{d}_{i}\big)^{n}_{i=1}$ & $\Re^3$ & Native color encoding of the camera system response values of training spectra. \\ \hline
	$\big(\mathbf{v}_{i}\big)^{n}_{i=1}$ & $\Re^3$ & Scaled white-balanced radiometrically linear camera system response values for training spectra. \\ \hline
	$\mathbf{h}_s$ & $\Re^m$ & Spectral power distribution of scene illumination source. \\ \hline
	$\mathbf{w}_s$ & $\Re^3$ & CIE XYZ tristimulus values of the scene illumination source. \\ \hline
	$\mathbf{h}_w$ & $\Re^m$ & Spectral power distribution of scene adopted white. \\ \hline	
	$\mathbf{w}_w$ & $\Re^3$ & CIE XYZ tristimulus values of the scene adopted white. \\ \hline
	$\mathbf{M}$ & $\Re^{3\times3}$ & Matrix converting ACES RGB relative exposure values to CIE XYZ tristimulus values. \\ \hline
	$\mathbf{w}$ & $\Re^3$ & CIE XYZ tristimulus values of ACES unity neutral. \\ \hline
	$\mathbf{X}$ & $\Re^{m\times3}$ & CIE 1931 2$^\circ$ Standard Observer color matching functions. $\mathbf{X}_\mathrm{Y}$ is the spectral luminous efficiency function. \\ \hline
	$\mathbf{b}$ & $\Re^3$ & Camera system white balancing and scaling factors. \\ \hline
	$\mathbf{B}$ & $\Re^{3\times3}$ & Matrix converting white-balanced scaled camera system RGB response values to ACES RGB relative exposure values. \\ \hline
	$\mathbf{A}_{Bradford}$ & $\Re^{3\times3}$ & Bradford matrix for chromatic adaptation, with value \\
	& & 
		$\begin{bmatrix*}[r]
			0.8950 & 0.2664 & -0.1614 \\
			-0.7502 & 1.7135 & 0.0367 \\
			0.0389 & -0.0685 & 1.0296 
		\end{bmatrix*}$ \\ \hline
	$\mathbf{A}_{CAT02}$ & $\Re^{3\times3}$ & CAT02 matrix for chromatic adaptation, with value \\
	& &
		$\begin{bmatrix*}[r]
			0.7328 & 0.4296 & -0.1624 \\
			-0.7036 & 1.6975 & 0.0061 \\
			0.0030 & 0.0136 & 0.9834 
		\end{bmatrix*}$ \\ \hline
	$\mathbf{A}_{re-illum}$ & $\Re^{3\times3}$ & Example matrix for re-illumination\footnote{Both the contents of this matrix and the method by which it is calculated may be protected by patent(s) including but not limited to U.S. Patent No. 7,298,892 (Inventors: Spaulding, Kevin E.; Woolfe, Geoffrey J.; and Giorgianni, Edward J.).}, with value \\
	& &
		$\begin{bmatrix*}[r]
			1.6160 & -0.3591 & -0.2569 \\
			-0.9542 & 1.8731 & 0.0811 \\
			0.0170 & -0.0333 & 1.0163 
		\end{bmatrix*}$ \\ \hline
\end{tabularx}
}

\section{Functions}
\label{sec:functions}
In the table and equations below, functions are indicated by a (possibly subscripted) lower case letter or word, followed by comma-separated arguments enclosed in parentheses. Arguments represented by italic lower case letters are scalars, those represented by lower case bold face letters are vectors, and those represented by upper case bold face letters are matrices.

\begin{tabularx}{\textwidth}{|l|l|X|}
	\hline
	\bfseries{Function} & \bfseries{Domain} & \bfseries{Description} \\ \hline
	$\mathrm{f_{CAM}}(\mathbf{x}, \mathbf{w})$ & $\Re^3\rightarrow\Re^3$ & Return color appearance correlates for tristimulus values $\mathbf{x}$ under adopted white $\mathbf{w}$ suited for measurement of color difference. \\ \hline
	$\min(\mathbf{v})$ & $\Re^m\rightarrow\Re$ & Return the minimum element of vector $\mathbf{v}$. \\ \hline
	$\gamma(\mathbf{c})$ & $\Re^3\rightarrow\Re^3$ & Decode camera image data code values $\mathbf{c}$ to normalized radiometrically linear camera system exposures. \\ \hline
	$\psi(\mathbf{A}, \mathbf{w})$ & $\Re^3\rightarrow\Re^3$ & Create a re-illumination matrix which transforms tristimulus values to an alternative color space using matrix $\mathbf{A}$, multiplies transformed values by the reciprocal of transformed white $\mathbf{w}$, and inverse transforms the scaled transformed values to their original color space using the inverse of matrix $\mathbf{A}$. \\
	& & $\psi(\mathbf{A}, \mathbf{w})=\mathbf{A}^{-1} \begin{bmatrix*}
			\rho & 0 & 0 \\
			0 & \gamma & 0 \\
			0 & 0 & \beta 
		\end{bmatrix*}\mathbf{A}$ \\
	& & with \\
	& & $\begin{bmatrix*}
			\rho \\
			\gamma \\
			\beta 
		\end{bmatrix*} = \dfrac{1}{\mathbf{Aw}}$ \\ \hline	
	$\phi(\mathbf{A}, \mathbf{w_1}, \mathbf{w_2})$ & $\Re^3\rightarrow\Re^3$ & Create a chromatic adaptation matrix which transforms tristimulus values to an alternative color space using matrix $\mathbf{A}$, scales transformed values by ratio of transformed white $\mathbf{w_2}$ to transformed white $\mathbf{w_1}$, and inverse transforms the scaled transformed values to their original color space using the inverse of matrix $\mathbf{A}$. \\
	& & $\phi(\mathbf{A}, \mathbf{w_1}, \mathbf{w_2})=\mathbf{A}^{-1} \begin{bmatrix*}
			\rho & 0 & 0 \\
			0 & \gamma & 0 \\
			0 & 0 & \beta 
		\end{bmatrix*}\mathbf{A}$ \\
	& & with \\
	& & $\begin{bmatrix*}
			\rho \\
			\gamma \\
			\beta 
		\end{bmatrix*} = \dfrac{\mathbf{Aw}_2}{\mathbf{Aw}_1}$ \\
	& & For equal whites, $\phi$ shall be the identity function: \\
	& & $\mathbf{w}_1=\mathbf{w}_2\Rightarrow\phi(\mathbf{A}, \mathbf{w_1}, \mathbf{w_2})=\mathbf{I}$ \\ \hline	
	$\min(\mathbf{v})$ & $\Re^m\rightarrow\Re$ & Return the minimum element of vector $\mathbf{v}$. \\ \hline
	$\mathrm{clip}(\mathbf{v})$ & $\Re^m\rightarrow\Re^m$ & Clip channel values in vector $\mathbf{v}$ to [1, 1, 1], or extrapolate the channels clipped on capture from the unclipped channels. \\ \hline
\end{tabularx}

\section{Operations}
\label{sec:operations}
In the table and equations below, lower case italic letters ($m$) represent scalars, lower case bold face letters ($\mathbf{h}$) represent vectors, and upper case bold face letters ($\mathbf{M}$) represent matrices.


\begin{tabularx}{\textwidth}{|l|l|X|}
	\hline
	\bfseries{Operation} & \bfseries{Domain} & \bfseries{Description} \\ \hline
	$\mathbf{v} + \mathbf{w}$ & $\Re^m\rightarrow\Re^m$ & Element-wise addition of vectors in $\Re^m$. \\ \hline
	$\mathbf{v} * \mathbf{w}$ & $\Re^m\rightarrow\Re^m$ & Element-wise multiplication of vectors in $\Re^m$. \\ \hline
	$\mathbf{v} / \mathbf{w}$ & $\Re^m\rightarrow\Re^m$ & Element-wise division of vectors in $\Re^m$, $w_i>0$. \\ \hline
	$\mathbf{v}\mathbf{w}$ & $\Re^m\rightarrow\Re$ & Dot product of vectors in $\Re^m$, $\mathbf{vw}=\displaystyle\sum_{i}^{m} v_iw_i$. \\ \hline
	$\mathbf{M}\mathbf{v}$ & $\Re^{nxm},\Re^m\rightarrow\Re^m$ & Matrix-vector multiplication $\mathbf{v}\ni\Re^m$, $\mathbf{M}\ni\Re^{nxm}$. \\ \hline
	$\mathbf{M}_{\mathrm{T}}$ & $\Re^{nxm}\rightarrow\Re^{mxn}$ & Transposition of matrix $\mathbf{M}$. \\ \hline
\end{tabularx}

\section{Constant symbol values}
\subsection{Compute CIE XYZ tristimulus values of the ACES unity neutral}
The CIE XYZ tristimulus values $\mathbf{w}$ of the ACES unity neutral will be the same for all IDTs, as the elements involved in its computation do not include scene- or camera-system-specific data; the values are computed by post-multiplication of $\mathbf{M}$ (the ACES RGB to CIE XYZ conversion matrix defined in Section 4.1.1 of the ACES document) by the ACES unity neutral:

$\mathbf{M} =
    \begin{bmatrix*}[r]
		0.9525523959 & 0 & 0.0000936786 \\
		0.3439664498 & 0.7281660966 & -0.0721325464 \\
		0 & 0 & 1.0088251844
    \end{bmatrix*}$

$\mathbf{w} =\mathbf{M}\begin{bmatrix} 1 \\ 1 \\ 1 \\ \end{bmatrix}$

\section{Computation}
\subsection{Compute (if required) spectral radiances of illuminated training spectral reflectances}
\label{sec:compstart}
When the training colors $\mathbf{t}_i$ are expressed as spectral radiances, they may be used directly in the equations below. When they are expressed as spectral reflectances $\mathbf{r}_i$ under a scene adopted white with spectral power distribution $\mathbf{h}_w$ the resulting spectral radiances are calculated as 

$\mathbf{t}_i = \mathbf{h}_w*\mathbf{r}_i$

\subsection{Compute CIE XYZ tristimulus values of training colors spectral radiances}
The training color CIE XYZ tristimulus values are calculated as 

$\mathbf{x}_i=\mathbf{X}^{\mathrm{T}}\mathbf{t}_i$

\subsection{Compute CIE XYZ tristimulus values of scene adopted white}
The scene adopted white CIE XYZ tristimulus values are calculated as 

$\mathbf{w}_w=\mathbf{X}^{\mathrm{T}}\mathbf{h}_w$

\subsection{Adjust training color CIE XYZ tristimulus values to compensate for difference between scene adopted white chromaticity and ACES neutral chromaticity}
\label{sec:chromaticadaptation}
Unless the scene adopted white happens to have the chromaticity of ACES neutrals (which have the chromaticity of CIE Standard Illuminant D60), some transformation is required to map the scaled white-balanced camera system response values that would result from the capture of the scene adopted white to the ACES unity neutral.

The training color CIE XYZ values are transformed to a balancing color space, scaled, then transformed back to CIE XYZ. When this balancing color space is structured around the cone responses of the human visual system (HVS), the scaling transform models chromatic adaptation; when the balancing color space is one of a class of color spaces particularly optimized for color adjustment, the scaling models re-illumination.


Three possible matrices representing a white balance transform are given in this section. The first and second model chromatic adaptation. The third models re-illumination.

\subsubsection{Alternatives 1 and 2: chromatic adaptation}
The training color CIE XYZ values are chromatically adapted such that neutral reflectors illuminated by the scene adopted white are encoded with the chromaticities of ACES neutrals. Two matrices are provided in this document, one defining the Bradford chromatic adaptation transform, and one defining the CAT02 transform.

$\mathbf{x}'_i = \phi(\mathbf{A}_{Bradford},\mathbf{w}_w,\mathbf{w})\mathbf{x}_i$ or $\mathbf{x}'_i = \phi(\mathbf{A}_{\mathit{CAT02}},\mathbf{w}_w,\mathbf{w})\mathbf{x}_i$


\subsubsection{Alternative 3: re-illuminate white-balanced ACES RGB relative exposure values for ACES neutral chromaticity}
The training color CIE XYZ values are scaled such that the scene adopted white CIE XYZ values are mapped to the ACES unity neutral CIE XYZ values and the white-balanced camera system exposure values are left unchanged.

$\mathbf{x}'_i = \phi(\mathbf{A}_{re-illum},\mathbf{w}_w)\mathbf{x}_i$

\subsection{Compute camera system white balance factors}
The camera system white balance factors map and normalize the linearized camera system responses corresponding to the capture of a perfect reflecting diffuser illuminated by the scene adopted white to camera system RGB values [1, 1, 1].

$\mathbf{b}=\dfrac{1}{\mathbf{C}^{\mathrm{T}}\mathbf{h}_w}$

\subsection{Compute white-balanced linearized camera system response values of training \\*colors}
The camera system spectral responsivity functions are post-multiplied by the training spectral radiances and scaled by the camera system white balance factors to produce the white-balanced, linearized, and normalized camera system response values for the set of training colors.

$\mathbf{v}_i=\mathbf{b}*\mathbf{C}^{\mathrm{T}}\mathbf{t}_i$

\subsection{Create initial values for the unoptimized IDT matrix entries}
\label{sec:compmat}
The matrix $\mathbf{B}$ is made up of 6 free parameters $\big(\beta_i\big)^{6}_{i=1}$.

$   \mathbf{B} =
    \begin{bmatrix}
		\beta_1 & \beta_2 & 1-\beta_1-\beta_2 \\
		\beta_3 & \beta_4 & 1-\beta_3-\beta_4 \\
		\beta_5 & \beta_6 & 1-\beta_5-\beta_6
    \end{bmatrix}$

Although other methods are possible, this document recommends the use of linear regression between the ACES RGB relative exposure values $\mathbf{x}'_i$ calculated in Section 4.7.3 and the white balanced camera system response values for the training colors $\mathbf{v}_i$ determined in Section 4.7.6.

The constraint that each row of matrix entries sums to unity can be formulated into the regression\footnote{Finlayson and Drew: White-point preserving color correction, Proceedings of the Fifth Color Imaging Conference, 1997.} or can be enforced by scaling the parameters obtained from unconstrained regression (although the latter gives only an approximation of the true solution).

\subsection{Select a cost function for error minimization}
\label{sec:compcostfn}
The cost function sums the distances between the target values and the transformed camera system response values after both sets of values have been transformed to a suitable color appearance space. The merit function is

$\chi^2=\displaystyle\sum_{i}^{n}\begin{Vmatrix}\mathbf{f}_{\mathrm{CAM}}(\mathbf{x}'_i,\mathbf{w})-\mathbf{f}_{\mathrm{CAM}}(\mathbf{MBv}_i,\mathbf{w})\end{Vmatrix}$
 
Choices for a suitable color appearance space for $\mathbf{f}_{\mathrm{CAM}}$ include

\begin{itemize}
	\item Nonlinear, neutral preserving RGB
	\item CIE L*u*v*
	\item CIE L*a*b*
	\item CIECAM

\end{itemize}

\note{Other color spaces may also be used.}

\note{More complex cost functions may also be used to weight the importance of particular training spectra, or to weight particular aspects of the approximation of the RICD's capture of training spectra (hue angle, luminance, colorfulness) more heavily than others, or in other ways chosen by the IDT author.}

\subsection{Find final matrix values minimizing cost function}
\label{sec:compend}
An iterative algorithm is used to minimize the cost function with respect to the parameters $\beta$. Existing implementations of this IDT calculation use the nonlinear least squares regression from MATLAB's Optimization Toolbox, or the \textbf{\texttt{optim}} function in R, or the \textbf{\texttt{FindMinimum}} function in Mathematica.


\subsection{Application}
\label{sec:application}
The IDT converts image data to ACES RGB relative exposure values $\mathbf{a}$ as follows:


$\mathbf{a} = k \mathbf{B} \mathrm{clip}\left(\dfrac{\mathbf{b}*\gamma(\mathbf{d})}{\min(\mathbf{b})}\right)$

where $\gamma$, $\mathbf{b}$, $\min()$, clip$()$ and $\mathbf{B}$ are as defined in Sections \ref{sec:symbols} and \ref{sec:functions}; $\mathbf{d}$ is an image data code value vector; and $k$ is the factor that results in a nominally ``18\% gray'' object in the scene producing ACES values [0.18, 0.18, 0.18].


\subsubsection{Application using CTL}
A CTL function taking a normalized linear camera system RGB code value and returning an ACES RGB relative exposure value could be implemented as follows:

\cleardoublepage
\begin{lstlisting}
// D200 IDT
// D55 illuminant
// camSPECS-measured D200 spec sens
// Woolfe/Spaulding/Giorgianni space for reconciliation of
// scene adopted white chromaticity and ACES neutral chromaticity
// (cf. U.S. Patent No. 7,298,892).

float
min(float a, float b) {
  if (a < b)
    return a;
  else
    return b;
}

float
clip(float v) {
  return min(v, 1.0);
}

void
main
(   input varying float rIn,
    input varying float gIn,
    input varying float bIn,
    input varying float aIn,
    output varying float rOut,
    output varying float gOut,
    output varying float bOut,
    output varying float aOut)
{

const float b[] = { 0.00102373, 0.000570585, 0.000797976 };

const float B[][] = { {  0.674849,    0.231105,  0.0940463 },
                      {  0.0431725,   1.06981,  -0.112982  },
                      {  0.0312748,  -0.151054,  1.11978   } };

const float min_b = min(b[0], min(b[1], b[2]));
const float e_max = 1.0;
const float k     = 1.0;

float clippedRGB[3];
clippedRGB [0] = clip((b[0] * rIn) / (min_b * e_max));
clippedRGB [1] = clip((b[1] * gIn) / (min_b * e_max));
clippedRGB [2] = clip((b[2] * bIn) / (min_b * e_max));

rOut = k * (B[0][0] * clippedRGB[0] + B[0][1] * clippedRGB[1] + B[0][2] * clippedRGB[2]);
gOut = k * (B[1][0] * clippedRGB[0] + B[1][1] * clippedRGB[1] + B[1][2] * clippedRGB[2]);
bOut = k * (B[2][0] * clippedRGB[0] + B[2][1] * clippedRGB[1] + B[2][2] * clippedRGB[2]);

}
\end{lstlisting}



\begin{appendices}
	\appendixchapter{Encoding of negative values}{i}
\label{appendixA}

Very small ACES scene referred values below $7\,^1/_4$ stops below 18\% middle gray are encoded as negative ACEScc values. These values should be preserved per the encoding in \autoref{sec:ACEScc} so that all positive ACES values are maintained.

When ACES values are matrixed into the smaller ACEScc color space, colors outside the ACEScc gamut can generate negative values even before the log encoding. If these values are clipped, a conversion back to ACES will not restore the original colors. A specific method of preserving negative values produced by the transformation matrix has not been defined in part to help ease adoption across various color grading systems that have different capabilities and methods for handling negative values. Clipping these values has been found to have minimal visual impact when viewed through the Reference Rendering Transform (RRT) and an appropriate Output Device Transform (ODT) on currently available display technology. However to preserve creative choice in downstream processing and to provide the highest quality archival master, developers implementing ACEScc encoding are encouraged to adopt a method of preserving negative values so that a conversion from ACES to ACEScc and back can be made lossless. Alternatively, a gamut mapping algorithm may be applied to minimize hue shifts resulting from clipping negative ACEScc values. Specific methods for handling negative values may be added to the ACEScc specification in the future.
	\appendixchapter{Application of ASC CDL parameters to ACEScc image data}{i}
\label{appendixB}

American Society of Cinematographers Color Decision List (ASC CDL) slope, offset, power, and saturation modifiers can be applied directly to ACEScc image data. ASC CDL color grades created on-set with ACESproxy images per the ACESproxy specification will reproduce the same look when applied to ACEScc images. ACEScc images however aren’t limited to the ACESproxy range. To preserve the extended range of ACEScc values, no limiting function should be applied with ASC CDL parameters. The power function, however, should not be applied to any negative ACEScc values after slope and offset are applied. Slope, offset, and power are applied with the following function.

\begin{gather*} 
    ACEScc_{out} = \left\{ 
    \begin{array}{l r }
        ACEScc_{in} \times slope + offset; & \quad ACEScc_{slopeoffset} \leq 0 \\
        (ACEScc_{in} \times slope + offset)^{power}; & \quad ACEScc_{slopeoffset} > 0 \\
    \end{array} \right. \\ 
    \\
    \begin{array}{l}
    \text{Where:}\\
    ACEScc_{slopeoffset} = ACEScc_{in} \times slope + offset
    \end{array}
\end{gather*}

ASC CDL Saturation is also applied with no limiting function:

\begin{gather*}
    luma = 0.2126 \times ACEScc_{red} + 0.7152 \times ACEScc_{green} + 0.0722 \times ACEScc_{blue} \\
    \begin{aligned}
        ACEScc_{red} &= luma + saturation \times (ACEScc_{red} - luma) \\
        ACEScc_{green} &= luma + saturation \times (ACEScc_{green} - luma) \\        
        ACEScc_{blue} &= luma + saturation \times (ACEScc_{blue} - luma) \\ 
    \end{aligned}
\end{gather*}
    
\end{appendices}

\end{document}