% Section Start
\section[P3-DCI]{\shortName{\id}}
\label{sec:odt-details-\id}

%% Summary
\subsection{Summary}
\label{subsec:summary-\id}

It is common in the digital intermediate process (DI) to color correct motion pictures and episodic television shows while displaying the images using a DCI compliant digital cinema projector. DCI compliant digital cinema projectors have a simplified setup using a projector configuration file (PCF) that contains all the relevant projector settings and can often be loaded at the press of a button. The color calibration the projector is specified in the PCF using a Target Color Gamut Document (TCGD) that includes aims of the calibrated display color primaries and white point chromaticity.  The most common PCF used in motion picture and television production is the ``DCI-P3'' PCF. Using this PCF, the projector will be configured such that equal red, green, and blue projector code values will produce the chromaticity \whitepoint{dci} on the screen. When using a the projector configured with the DCI-P3 PCF, or any PCF based on the DCI-P3 PCF, it is recommended that the ACES 1.0 ODT with the transformID \transformID{\id} be used.

%% Transform Identifiers
\subsection{Transform Identifiers} 
\label{subsec:odt-ident-\id}
\idTable{1.5}{3}

%% Recommended Display Setup
\subsection{Recommended Display and Setup}
\label{subsec:setup-\id}

\begin{table}[ht!]
    \centering
        \begin{tabular}{|p{1.5in}|p{3in}|}
            \hline
            \textbf{Parameter} 		& 	\textbf{Setting} 				 		\\ \hline
            Display Type 			&	Digital Cinema compliant projector 		\\ \hline
            Display Dynamic Range 	& 	$\geq$ 2,000:1 to $\sim$10,000:1 		\\ \hline
            Display Max Luminance 	& 	\nits{48}								\\ \hline
            TCGD 					& 	?? DCI P3 lookup name?? 				\\ \hline %TODO Find P3 DCI TCGD Filename
            Signal 					&	RGB 4:4:4 Full range 					\\ \hline
            Viewing Environment 	& 	Dark 									\\ \hline
            Bit Depth 				& 	12-bit 									\\ \hline 
    	\end{tabular}
    \caption{Display Setup: \protect\shortName{\id}} 
    \label{tab:setup-\id}
\end{table}

%% Notes
\subsection{Notes}
\label{subsec:notes-\id}

Using the ``DCI-P3'' PCF, the projector will be configured such that equal red, green, and blue display code values will produce the chromaticity \whitepoint{dci} on the screen. However, the \transformID{\id} transform is configured such that neutral ACES source file values (ACES \rgbequal{}) will produce non-equal projector code values. The chromaticity of produced on screen by those non-equal projector code values will be \whitepoint{aces}. (aka D$_{60}$) 

It's important to note that the image on projection screen may look distinctly less green than some workflows that utilize a projector setup with the ``DCI-P3'' PCF. This will also be reflected on the color corrector scopes when neutral ACES values sent through the \transformID{\id} transform. (Figure \ref{fig:acesSource-p3dci}, \ref{fig:hist-p3dci}, \ref{fig:parade-p3dci}, \ref{fig:wf-p3dci}, \ref{fig:vect-p3dci}) For instance, neutral ACES values processed through \transformID{\id} will not have equal levels on the waveform, nor will they land in the middle of the vector scope. This behavior was intentional. The image may also have a distinctly magenta cast on a computer monitor such as the on used for the color corrector user interface if that monitor is calibrated to a D$_{65}$ white point. (Figure \ref{fig:cv-p3dci}) Although not noted in the name of this ODT, the mimics the behavior found in other ODTs include in ACES 1.0 and labeled ``D60 sim''. Due to this ``D60 sim'' behavior the maximum output screen luminance of neutral ACES values will b slightly less than the maximum calibration luminance (e.g.~\nits{48}) produced by the maximum equal projector code values (CV \rgbequalone{}) 

When using the correct projector setup and corresponding ODT, the image on the projector screen will match nearly exactly in Application \ref{sec:odt-details-p3dci} an Application \ref{sec:odt-details-p3d60} 

\odtScreenshots{Projector code values as displayed on a D$_{65}$ calibrated computer monitor}

%% Test Values
\subsection{Test Values}
\testValuesSubSec{}

